\documentclass{beamer}
\usepackage[slovene]{babel}
\usepackage[utf8]{inputenc}
\usepackage{lmodern}
%\usepackage[T1]{fontenc}
\setlength\parindent{0pt}
\usetheme{Warsaw}

\newcommand{\blue}[1]{\textcolor[rgb]{0,0,1}{#1}}
\newcommand{\red}[1]{\textcolor[rgb]{1,0,0}{#1}}
\newcommand{\orange}[1]{\textcolor[rgb]{1,0.41,0.13}{#1}}
\newcommand{\green}[1]{\textcolor[rgb]{0.2,0.8,0.2}{#1}}
\newcommand{\grey}[1]{\textcolor[rgb]{0.75,0.75,0.75}{#1}}

\newtheorem{izrekparam}{Izrek parametričnosti}

\title{Zastonjski izreki}
\author{Domen Močnik\\
Matej Aleksandrov}
\date{Ljubljana, 8. januar 2015}

\begin{document}

\begin{frame}
	\titlepage
	\end{frame}
	
	
	\begin{frame}
	\frametitle{Osnovna ideja}
	%Osnovna ideja:
	%\\
	\begin{center}
	Tip polimorfne funkcije $\rightarrow$ izrek
	\end{center}
	\pause
	Primer:
	\\
	$$f: \forall X .\mbox{ } X^* \rightarrow X^*$$
	\pause
	Potem za vsak $a: A \rightarrow A'$ velja izrek:
	$$  a^* \circ f_{A} = f_{A'} \circ a^* $$
	\end{frame}
	
	\begin{frame}
	\frametitle{Parametričnost}
		Funkcijo $f:A \to A'$ lahko predstavimo kot relacijo $\mathcal{A} \subseteq A \times A'$, kjer velja $(x,x')\in \mathcal{A} \Leftrightarrow  f(x)=x'$.\\ \pause
		Konstruktorji tipov:\pause
		\begin{itemize}
		\item $\mathcal{A}\times \mathcal{B} \subseteq (A \times B) \times (A' \times B')$:
		 \begin{center} $((x,y),(x',y') \in \mathcal{A}\times \mathcal{B} \Leftrightarrow (x,x') \in \mathcal{A} \mbox{ in } (y,y') \in \mathcal{B} $ \end{center} \pause
		\item $\mathcal{A}^* \subseteq A^* \times A'^*$:
		$$ ([x_1, \ldots, x_n],[x'_1, \ldots, x'_n]) \in \mathcal{A}^* \Leftrightarrow (x_1,x'_1) \in \mathcal{A},  \ldots  (x_n,x_n') \in \mathcal{A} $$ \pause
		\vspace{-0.5cm}
		\item $\mathcal{A}\to \mathcal{B} \subseteq (A \to B) \times (A' \to B')$:
		 \begin{center} $ (f,f') \in \mathcal{A}\to \mathcal{B} \Leftrightarrow  \forall (x,x') \in \mathcal{A}.\mbox{ } (f(x),f'(x')) \in \mathcal{B} $  \end{center} \pause
		\item $\forall \mathcal{X}.\mbox{ } \mathcal{F}(\mathcal{X}) \subseteq \forall X.\mbox{ } F(X) \times \forall X.\mbox{ } F'(X) $:
		 \begin {center}$ (g,g') \in \forall \mathcal{X}.\mbox{ } \mathcal{F}(\mathcal{X})  \Leftrightarrow  \forall \mathcal{A} \subseteq (A,A').\mbox{ } (g_A,g'_{A'}) \in \mathcal{F}(\mathcal{A})$  \end{center}
		\end{itemize}
\pause
	Konstantne tipe (Int, Bool, Char,...) predstavimo z identično relacijo.
	\end{frame}
	
	\begin{frame}
	\begin{izrekparam}
	Naj bo $\mathcal{T} \subseteq T \times T$ relacija, ki pripada zaprtemu tipu $T$ in $t\in T$. Potem je $(t,t) \in \mathcal{T}$.
	\end{izrekparam}
	\pause
	\vspace{1cm}
	Začetni primer:
	\begin{eqnarray*}
	& f: \forall X .\mbox{ } X^* \rightarrow X^* \\ \pause
	\\
	\Rightarrow & (f,f) \in \forall \mathcal{X} .\mbox{ } \mathcal{X}^* \rightarrow \mathcal{X}^* \\ \pause
	\Rightarrow & \forall \mathcal{A} \subseteq (A,A'). \mbox{ } (f_A,f'_{A'}) \in \mathcal{A}^* \to \mathcal{A}^* \\ \pause
	\Rightarrow & \forall \mathcal{A} \subseteq (A,A').\mbox{ }  \forall (xs,xs') \in \mathcal{A}^* . \mbox{ } (f_A (xs),f'_{A'}( xs')) \in \mathcal{A}^* \\ \pause
	\Rightarrow & \forall a : A \to A' . \mbox{ } \forall xs \in A^* . \mbox{ } a^*(xs)=xs' \Rightarrow a^*(f_A(xs)) = f_{A'}(xs') \\ \pause
	\Rightarrow & \forall a : A \to A' .  \mbox{ } a^* \circ f_{A} = f_{A'} \circ a^* \\
	\end{eqnarray*}
	
	\end{frame}	


\begin{frame}
\frametitle{Podatkovne strukture}
%\framesubtitle{Tip}
% Iz katerih podatkovnih struktur gradimo Zastonjske izreke?
Tip:\\
\texttt{
{\footnotesize type TypeVariable = String}
\begin{block}{\texttt{data Type}}
\blue{\textbf{TypeVar}} \red{TypeVariable} 
\hfill{\small\sffamily spremenljivka} \\
$\qquad${\small a, b, x, y, ...} \\
\blue{\textbf{TypeConst}} \red{TypeVariable}
\hfill{\small\sffamily konstantni tip}\\
$\qquad${\small Bool, Int, Double, Char, ...} \\
\blue{\textbf{TypeFun}} \red{Type Type}
\hfill{\small\sffamily funkcija} \\
$\qquad${\small a -> b, x -> Bool, [Int] -> Int, ...} \\
\blue{\textbf{TypeList}} \red{Type}
\hfill{\small\sffamily seznam} \\
$\qquad${\small [a], [Int], [x -> y], ...} \\
\end{block}
}
\end{frame}

\begin{frame}
\frametitle{Podatkovne strukture}
%\framesubtitle{Izraz}
% Izrazi pripadajo tipom.
% Konstantni izrazi so npr True, 2, "whoa", 'b', 3.14, ...
% Lahko so spremenljivke.
% Lahko je aplikacija izraza na izraz, npr. uporaba funkcije na argumentu.
Izraz:\\
\texttt{
{\footnotesize type TermVariable = String}
\begin{block}{\texttt{data Term}}
\blue{\textbf{TermVar}} \red{TermVariable}
\hfill{\small\sffamily spremenljivka} \\
$\qquad${\small x, y, u, v, ...} \\
\blue{\textbf{TermApp}} \red{Term Term}
\hfill{\small\sffamily aplikacija izraza na izraz} \\
$\qquad${\small f x, g (f y), ...} \\
\end{block}
}
\end{frame}

\begin{frame}
\frametitle{Podatkovne strukture}
%\framesubtitle{Formula}
Formula:\\
\texttt{
{\footnotesize type RelationVariable = String}
\begin{block}{\texttt{data Formula}}
\blue{\textbf{ForallRelations}} \red{RelationVariable}\\
$\phantom{\textbf{ForallRelations}}$ (\red{TypeVariable}, \red{TypeVariable})\\
$\phantom{\textbf{ForallRelations}}$ \red{Formula}\\
$\qquad${\small $\forall\, \texttt{t}_1, \texttt{t}_2 \in \textsc{Types},\; \texttt{R} \subseteq \texttt{t}_1\times \texttt{t}_2\,$. (\textit{formula})}\\
\blue{\textbf{ForallVariables}} \red{TermVariable}
\red{Type} \red{Formula}\\
$\qquad${\small $\forall$ n :: Int . (\textit{formula})}\\
$\qquad${\small $\forall$ g :: $t_1$ -> $t_3$ . (\textit{formula})}\\
\blue{\textbf{ForallPairs}} (\red{TermVariable}, \red{TermVariable})\\
$\phantom{\textbf{ForallPairs}}$ \red{RelationVariable}\\
$\phantom{\textbf{ForallPairs}}$ \red{Formula}\\
$\qquad${\small $\forall$ (x, y) $\in$ R. (\textit{formula})}\\
$\qquad${\small $\forall$ (u, v) $\in$ [R -> S]. (\textit{formula})}\\
\end{block}
}
\end{frame}

\begin{frame}
\frametitle{Podatkovne strukture}
%\framesubtitle{Formula}
\texttt{
\begin{block}{\texttt{data Formula}}
$\vdots$\\
\blue{\textbf{IsMember}} (\red{Term}, \red{Term}) \red{RelationVariable}\\
$\qquad${\small (u, v) $\in$ R}\\
$\qquad${\small (g x, g y) $\in$ S}\\
\blue{\textbf{Implication}} \red{Formula Formula}\\
$\qquad${\small (\textit{formula}${}_1$) $\Rightarrow$ (\textit{formula}${}_2$)}\\
\blue{\textbf{Equivalence}} \red{Term Term}\\
$\qquad${\small x = y}\\
$\qquad${\small f x = g y}\\
\end{block}
}
\end{frame}


\begin{frame}
Viri:
\begin{itemize}
\item P.~Walder, \emph{Theorems for Free!}, dostopno na \url{http://ttic.uchicago.edu/~dreyer/course/papers/wadler.pdf}.
\item J.~ Voigtlander, \emph{Free Theorems Involving Type Constructor Classes}, dostopno na \url{http://www.janis-voigtlaender.eu/papers/FreeTheoremsInvolvingTypeConstructorClasses.pdf}.
\item \emph{Automatic generation of free theoremss}, dostopno na \url{http://www-ps.iai.uni-bonn.de/cgi-bin/free-theorems-webui.cgi}.
\end{itemize}
\end{frame}

\end{document}




















\begin{comment}

\begin{frame}
\frametitle{Generiranje izreka}
\framesubtitle{Kreiranje osnovnih relacij}
\begin{itemize}
\item V vhodnem tipu poišče vse spremenljivke. \\ % Samo spremenljivke, ne konstantnih tipov.
\texttt{(a -> [b] -> a) -> Int -> c} $\quad\longmapsto\quad$ \texttt{a, b, c}
\item Vsaki spremenljivki priredi par tipovnih spremenljivk ter relacijsko spremenljivko.\\
Sestavi začetek formule:\\
$\quad\forall\, \texttt{t}_1, \texttt{t}_2 \in \textsc{Types},\; \texttt{R} \subseteq \texttt{t}_1\times \texttt{t}_2\,$.
\hfill\grey{$\leftarrow$ pripada \texttt{a}}\\
$\quad\quad\forall\, \texttt{t}_3, \texttt{t}_4 \in \textsc{Types},\; \texttt{S} \subseteq \texttt{t}_3\times \texttt{t}_4\,$.
\hfill\grey{$\leftarrow$ pripada \texttt{b}}\\
$\quad\quad\quad\forall\, \texttt{t}_5, \texttt{t}_6 \in \textsc{Types},\; \texttt{T} \subseteq \texttt{t}_5\times \texttt{t}_6\,$.
\hfill\grey{$\leftarrow$ pripada \texttt{c}}
\end{itemize}
\end{frame}

\begin{frame}
\frametitle{Generiranje izreka}
\framesubtitle{Pretvorba spremenljivk, konstantnih tipov ter seznamov}
\texttt{
\begin{block}{}
formula :: Term -> Term -> Type -> Formula
\end{block}
formula term${}_1$ term${}_2$ type = \blue{case} type \blue{of}
\begin{itemize}
\item TypeVar \green{a} $\longmapsto$ IsMember (term${}_1$, term${}_2$) \orange{R${}_a$}\\
$\quad${\footnotesize \orange{R${}_a$} {\sffamily osnovna relacija, ki pripada spremenljivki} \green{a}}
\item TypeConst \green{c} $\longmapsto$ Equivalence term${}_1$ term${}_2$ %po izreku o parametričnosti
\item TypeList t $\longmapsto$ IsMember (term${}_1$, term${}_2$) \orange{[Rel${}_t$]}\\
{\footnotesize
$\quad$\orange{[Rel${}_t$]} {\sffamily sestavljena relacija, ki pripada tipu t}\\
$\quad\;$\orange{Rel${}_t$} = \blue{case} t \blue{of}\\
$\quad\quad$TypeVar \green{a} $\rightarrow$ {\sffamily osnovna relacija, ki pripada sprem. }\green{a}\\
$\quad\quad$TypeConst \green{c} $\rightarrow$ \orange{Id${}_{\mathtt{c}}$}
$\quad$npr. \orange{Id${}_{\mathtt{Bool}}$}, \orange{Id${}_{\mathtt{Double}}$}, ...\\
$\quad\quad$TypeList ty $\rightarrow$ \orange{[} {\sffamily sestavljena relacija, ki pripada tipu }ty 
\orange{]}\\
$\quad\quad$TypeFun t${}_1$ t${}_2$ $\rightarrow$ ({\sffamily sestavljena relacija, ki pripada tipu }t${}_1$) \orange{->}\\
$\quad\quad$\phantom{TypeFun t${}_1$ t${}_2$ $\rightarrow$ }({\sffamily sestavljena relacija, ki pripada tipu }t${}_2$)
}
\end{itemize}
}
\end{frame}

\begin{frame}
\frametitle{Generiranje izreka}
\framesubtitle{Pretvorba funkcij}
\texttt{
\begin{block}{}
formula :: Term -> Term -> Type -> Formula
\end{block}
formula term${}_1$ term${}_2$ type = \blue{case} type \blue{of}
\begin{itemize}
\item TypeFun (TypeVar \green{a}) b $\longmapsto$ \\
\blue{let} (\green{x}, \green{y}) $\leftarrow$ generator \ 
{\footnotesize\sffamily \grey{generira spremenljivki za nova izraza}}\\
$\phantom{let\ }$ \orange{R${}_{\mathtt{a}}$} = {\sffamily osnovna relacija, ki pripada sprem. }\green{a}\\
$\phantom{let\ }$ nextFormula = formula\\
$\qquad\quad$ (TermApp term${}_1$ (TermVar \green{x}))\\
$\qquad\quad$ (TermApp term${}_2$ (TermVar \green{y})) \\ $\qquad\quad$ b \\
\blue{in\ } ForallPairs (\green{x}, \green{y})
\orange{R${}_{\mathtt{a}}$} nextFormula
\end{itemize}
}
\end{frame}

\begin{frame}
\frametitle{Generiranje izreka}
\framesubtitle{Pretvorba funkcij}
\texttt{
\begin{block}{}
formula :: Term -> Term -> Type -> Formula
\end{block}
formula term${}_1$ term${}_2$ type = \blue{case} type \blue{of}
\begin{itemize}
\item TypeFun (TypeConst \green{c}) b $\longmapsto$ \\
\blue{let} \green{z} $\leftarrow$ generator \ 
{\footnotesize\sffamily \grey{generira spremenljivko za nov izraz}}\\
$\phantom{let\ }$ nextFormula = formula\\
$\qquad\quad$ (TermApp term${}_1$ (TermVar \green{z}))\\
$\qquad\quad$ (TermApp term${}_2$ (TermVar \green{z})) \\ $\qquad\quad$ b \\
\blue{in\ } ForallVariables \green{z}
(TypeConst \green{c}) nextFormula
\end{itemize}
}
\end{frame}

\begin{frame}
\frametitle{Generiranje izreka}
\framesubtitle{Pretvorba funkcij}
\texttt{
\begin{block}{}
formula :: Term -> Term -> Type -> Formula
\end{block}
formula term${}_1$ term${}_2$ type = \blue{case} type \blue{of}
\begin{itemize}
\item TypeFun (TypeList ty) b $\longmapsto$ \\
\blue{let} (\green{u}, \green{v}) $\leftarrow$ generator \ 
{\footnotesize\sffamily \grey{generira spremenljivki za nova izraza}}\\
$\phantom{let\ }$ \orange{[R${}_{\mathtt{ty}}$]} = {\sffamily sestavljena relacija, ki pripada tipu } [ty]\\
$\phantom{let\ }$ nextFormula = formula\\
$\qquad\quad$ (TermApp term${}_1$ (TermVar \green{u}))\\
$\qquad\quad$ (TermApp term${}_2$ (TermVar \green{v})) \\ $\qquad\quad$ b \\
\blue{in\ } ForallPairs (\green{u}, \green{v})
\orange{[R${}_{\mathtt{ty}}$]} nextFormula
\end{itemize}
}
\end{frame}

\begin{frame}
\frametitle{Generiranje izreka}
\framesubtitle{Pretvorba funkcij}
$R\subseteq t_1\times t_2$, $S\subseteq t_3\times t_4$ $\Longrightarrow$\\
$R\rightarrow S =\{(g,h)\, |\, g :: t_1\rightarrow t_3, h :: t_2\rightarrow t_4\}$
\texttt{
%\begin{block}{}
%formula :: Term -> Term -> Type -> Formula
%\end{block}
%formula term${}_1$ term${}_2$ type = \blue{case} type \blue{of}
\begin{itemize}
\item TypeFun a b $\longmapsto$ \\
\blue{let} (\green{g}, \green{h}) $\leftarrow$ generator \ 
{\footnotesize\sffamily \grey{spremenljivki za novi funkciji}}\\
$\phantom{let\ }$ R = {\sffamily sestavljena relacija, ki pripada tipu } a -> b\\
$\phantom{let\ }$ leftFunctionType = {\sffamily levi funkcijski tip relacije } R\\
$\phantom{let\ }$ rightFunctionType = {\sffamily desni funkcijski tip relacije } R\\
$\phantom{let\ }$ fstFormula = formula\\
$\qquad\quad$ (TermVar \green{g}, TermVar \green{h}) a \\
$\phantom{let\ }$ sndFormula = formula\\
$\qquad\quad$ (TermApp term${}_1$ (TermVar \green{g}))\\
$\qquad\quad$ (TermApp term${}_2$ (TermVar \green{h})) \\ $\qquad\quad$ b \\
\blue{in\ } ForallVariables \green{g} leftFunctionType \$ \\
$\phantom{in\ }\quad$ ForallVariables \green{h} rightFunctionType \$ \\
$\phantom{in\ }\quad\quad$ Implication fstFormula sndFormula \\
\end{itemize}
}
\end{frame}

\end{comment}

\end{document}
